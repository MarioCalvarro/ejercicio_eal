\section{Ejercicio 41}
\begin{enun}
    Sea $K$ un cuerpo de característica $0$ y $a, b \in K$ tales que el polinomio $f\left( \mathrm{t} \right) := \mathrm{t}^4 + a \mathrm{t}^2 + b$ es irreducible en $K\left[\mathrm{t}\right]$.
    Hallar, en función de los valores de $a$ y $b$, el grupo de Galois de $f$ sobre $K$.
\end{enun}

\begin{sol}
    Sabemos\footnote{Subsección 3.c\cite{cuerpos}.} que la resolvente del polinomio será $g\left( \mathrm{t} \right) = \mathrm{t}^3 - 2a \mathrm{t}^2 + \left( a^2 - 4b \right)\mathrm{t}$ que es claramente reducible en $K\left[\mathrm{t}\right]$. También sabemos\footnote{Ejemplos VII.2.13 (3.1)\cite{anillos}.} que $\Delta\left( f \right) = 16b\left( 4b - a^2 \right)^2$. Observando entonces la tabla de la Proposición V.3.3\cite{cuerpos}, tenemos que el grupo de Galois de $f$ sobre $K$, $G_K\left( f \right)$, será isomorfo a una de las siguientes opciones: $\mathbb{Z}_2 \times \mathbb{Z}_2,\ \matheuler{D}_4$ ó $\mathbb{Z}_4$.

    Empecemos distinguiendo el caso en el que la raíz cuadrada (positiva) del discriminante pertenece a $K$. Esto es equivalente a que $\delta := \sqrt{16b\left( 4b - a^2 \right)^2} = 4 \sqrt{b} \left( 4b - a^2 \right) \in K$. Como el segundo término del producto lo cumple por definición, solo nos queda ver cuando se da para el primero. Trivialmente se dará si $\exists c \in K : b = c^2$. Si esto ocurre, observamos la tabla y vemos que $\boxed{G_K\left( f \right) \simeq \mathbb{Z}_2 \times \mathbb{Z}_2}$. En caso contrario, que es lo que asumiremos en lo sucesivo, tendremos que no existe tal $c$ o, dicho de otra manera, $b$ no es un cuadrado en $K$ y $G_K\left( f \right)$ será isomorfo a alguno de los otros dos grupos nombrados. Además, tendremos que $K\left( \delta \right) = K\left( \sqrt{b} \right)$. Veamos cuando se da cada caso.

    Estudiemos ahora la irreducibilidad de $f$ en $K\left( \delta \right)\left[ \mathrm{t} \right]$ según $a$ y $b$. Que $f$ sea reducible puede significar una de dos cosas:
    \begin{itemize}
        \item Que tenga una raíz $\alpha \in K\left( \delta \right)$.
        \item Que $f$ se descomponga como el producto de otros dos polinomios irreducibles de grado $2$ sobre $K\left( \delta \right)\left[ \mathrm{t} \right]$.
    \end{itemize}
    Sin embargo, la primera posibilidad la podemos descartar debido a que, si la asumimos como cierta, al ser $f$ irreducible en $K\left[ \mathrm{t} \right]$, tendríamos la siguiente desigualdad:\footnote{Transitividad del grado. Proposición I.1.6\cite{cuerpos}.}
    \[
    4 = \mathrm{deg}\left( f \right) = \mathrm{deg}\left( P_{K, \alpha} \right) = \left[ K\left( \alpha \right) : K \right] \le \left[ K\left( \delta \right) : K \right] = 2
    \]
    que es una contradicción.

    Veamos, pues, bajo que condiciones podemos descomponer $f$ como producto de polinomios de grado dos en $K\left( \delta \right)\left[ \mathrm{t} \right]$. Debido a que $f$ es un polinomio bicuadrado podemos calcular fácilmente sus raíces:
    \[
    u := \sqrt{\frac{-a + \sqrt{a^2 - 4b}}{2}}\qquad v := \sqrt{\frac{-a - \sqrt{a^2 - 4b}}{2}}
    \]
    y sus opuestos. Naturalmente, $u,v \in K_f$\footnote{$K_f$ denota el cuerpo de descomposición de $f$ sobre $K$.} lo que quiere decir, a su vez, que $\beta := \sqrt{a^2 - 4b} = 2u^2 + a \in K_f$. Por lo tanto, podemos factorizar $f$ como:
    \begin{align*}
        f\left( \mathrm{t} \right) &= \mathrm{t}^4 + a \mathrm{t}^2 + b\\
        &= \left( \mathrm{t}^2 + \frac{a}{2} \right)^2 - \frac{a^2 - 4b}{4}\\
        &= \left( \mathrm{t}^2 + \frac{a}{2} - \frac{\sqrt{a^2 - 4b}}{2} \right) \left( \mathrm{t}^2 + \frac{a}{2} + \frac{\sqrt{a^2 - 4b}}{2} \right)\\
        &= \left( t^2 + \frac{a}{2} - \frac{\beta}{2} \right)\left( \mathrm{t}^2 + \frac{a}{2} + \frac{\beta}{2} \right)
    \end{align*}
    Ahora, si $\beta \in K\left( \sqrt{b} \right)$, entonces, $f$ es reducible en $K\left( \sqrt{b} \right)\left[ \mathrm{t} \right]$ y $\boxed{G_K\left( f \right) \simeq \mathbb{Z}_4}$.

    Si suponemos ahora que $\beta \not\in K\left( \sqrt{b} \right)$, tendremos que $K\left( \sqrt{b} \right) \mid K$ y $K\left( \sqrt{a^2 - 4b} \right) \mid K$ son dos subextensiones distintas de grado $2$ de $K_f \mid K$, pero como $\mathbb{Z}_4$ solo tiene un subgrupo de índice $2$, por la primera parte del Teorema Fundamental de la teoría de Galois,\footnote{Teorema IV.2.5\cite{cuerpos}.} solo puede haber una subextensión de grado $2$. Como solo teníamos dos alternativas, $\boxed{G_K\left( f \right) \simeq \matheuler{D}_4}$.
\end{sol}
